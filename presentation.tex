\documentclass{beamer}
\usepackage[utf8]{inputenc}
\usepackage{pifont}
\newcommand{\cmark}{\ding{51}}%
\newcommand{\xmark}{\ding{55}}%

\begin{document}

\title{The Great Trade Collapse}
\subtitle{A Review of Bems et al 2013 for \\
  CSS695 Fall 2018}
\author{David Prentiss}
\institute{George Mason University}
\date{\today}

\frame{\titlepage}

\begin{frame}
  \frametitle{Intro and Conclusions}
  \begin{itemize*}
  \item Three main proximate causes and/or propagation mechanisms that account for the collapse of international trade:
  \item \textcolor{green}{\cmark} Changes in expenditure: 65-80\%
  \item \textcolor{green}{\cmark} Credit shocks: 15-20\%
  \item \textcolor{green}{\cmark} Inventory adjustments: up to 20\%
  \item \textcolor{red}{\xmark} Trade policy: 0\%
  \end{itemize*}
\end{frame}

\begin{frame}
  \frametitle{Stylized facts}
  \begin{center}
    \includegraphics[scale=0.31]{graph.png}
  \end{center}
\end{frame}

\begin{frame}
  \frametitle{Stylized facts}
  \begin{enumerate}
  \item The trade collapse stands out relative to post-WWII experience in terms of its abruptness, magnitude, and synchronization across countries.
  \item The trade collapse was asymmetric across sectors.
  \item The trade collapse occurred almost entirely on the intensive margin of trade, whether measured at the sector or firm level.
  \item The trade collapse mostly resulted from changes in the volume of trade, not price changes.
  \end{enumerate}
\end{frame}

\begin{frame}
  \frametitle{Framework}
  Constant elasticity of substitution import demand:
  \begin{equation*}
    d_{ij} = \left(\frac{\tau_{ij}p_i}{P_j}\right)^{-\sigma}D_j
  \end{equation*}
  \begin{itemize}
  \item \(d_{ij}\) is the aggregate real imports by country \(j\) from country \(i\),
  \item \(p_i\) is the price at the factory gate of the good from country \(i\),
  \item \(\tau_{ij}\) is an ad valorem cost of delivering goods from the factory gate in country \(i\) to final purchasers in country \(j\),
  \item \(P_j\) is the aggregate price index for final expenditure in country \(j\), and
  \item \(D_j\) is real aggregate expenditure by country \(j\) (expenditure).
  \end{itemize}
\end{frame}

\begin{frame}
  \frametitle{Framework}
  \begin{itemize}
    \item Hold trade costs \(\tau_{ij}\) constant and take the natural log.
      \item Log-linear import demand equation:
  \begin{equation*}
    \hat{d}_{ij} = - \sigma\left(\hat{p}_i-\hat{P}_j\right)-\hat{D}_j
  \end{equation*}
  \item \(\hat{d}\) is the CES-predicted change in imports.
  \item Decrease aggregate expenditure (\(\hat{D}_j<0\))?
  \item Increase in the relative prices of imports (\(\hat{p}_i-\hat{P}_j<0\))?
  \end{itemize}
\end{frame}

\begin{frame}
  \frametitle{Framework}
  \begin{itemize}
  \item Early research suggested this was too simple.
  \item Enter: ``the trade wedge'', \(\omega\)
  \begin{align*}
    \omega &\equiv\hat{m}_{ij}-\hat{d}_{ij} = \hat{m}_{ij} + \sigma\left(\hat{p}_i-\hat{P}_j\right)-\hat{D}_j \\
    \hat{m}_{ij} &= \beta_1\hat{\tau}_{ij}+\beta_2\left(\hat{p}_i-\hat{P}_j\right) + \beta_3\hat{D}_j \\
    \omega &= \beta_1\hat{\tau}_{ij}+\left(\beta_2-\sigma\right)\left(\hat{p}_i-\hat{P}_j\right) + \left(\beta_3-1\right)\hat{D}_j
  \end{align*}
    \item \(\hat{m}_{ij} \) is the actual change in imports.
    \item \(\beta_1\) is the elasticity of real imports with respect trade costs.
    \item \(\beta_2\) is the elasticity of relative prices.
    \item \(\beta_3\) is the elasticity of aggregate expenditure.
  \end{itemize}
\end{frame}

\begin{frame}
  \frametitle{Framework}
  The trade wedge:
  \begin{equation*}
    \omega = \beta_1\hat{\tau}_{ij}+\left(\beta_2-\sigma\right)\left(\hat{p}_i-\hat{P}_j\right) + \left(\beta_3-1\right)\hat{D}_j
  \end{equation*}
  \begin{enumerate}
  \item Increases in policy or nonpolicy trade costs.
  \item Factors that amplify the empirical elasticity of imports to final demand relative to calibrated CES elasticities.
  \item Multisector, asymmetric spending changes across sectors.
  \end{enumerate}
  \begin{itemize*}
  \item ``...we propose mapping explanations for the trade collapse into either (a) relative price changes or (b) changes in the trade wedge.''
  \end{itemize*}
\end{frame}

\begin{frame}
  \frametitle{The composition of expenditure changes}
  Asymmetric expenditure across sectors:
  \begin{itemize}
    \item Extend the benchmark framework to account for asymmetric expenditure
      across (2) sectors.
      \item The sectors are goods and services with only goods traded.
    \begin{equation*}
      \hat{m}=\left[ \sum_s \alpha^m(s)\left( \frac{\hat{d}(s)}{\hat{D}} \right) \right]\hat{D}
    \end{equation*}
    \item If sectors with relatively large expenditure change have relatively large import share, then the elasticity of imports to aggregate expenditure will be greater than one.
  \end{itemize}
\end{frame}

\begin{frame}
  \frametitle{The composition of expenditure changes}
  Input linkages across sectors and countries:
  \begin{itemize}
  \item Extend the model again to account for intermediate goods.
  \item With no intermediate goods imported, the model is the same as the
    previous model. (\(\beta_3>1\))
  \item If intermediate goods are imported, and the fall in expenditure is
    larger for goods than for services (assuming both fall), the
    elasticity of imports to aggregate expenditure is pushed back toward 1.
  \item Imported inputs tend to decrease the magnitude of the trade wedge.
  \item Similar result for trade between countries.
  \end{itemize}
\end{frame}

\begin{frame}
  \frametitle{The composition of expenditure changes}
  The evidence:
  \begin{itemize}
    \item A global input-output framework (forward model?) was fed observed
      observed sectoral expenditure changes to predict trade flows during the crisis.
      \item Another model (Ricardian) was used to break down the trade collapse
        into components due to changes in final expenditure, changes in
        productivity, changes in border trade frictions, and changes in trade
        deficits. (Model dependent, measurement device)
        \item Others used input-output tables to measure import demand, taking input linkages into account.
  \end{itemize}
\end{frame}

\begin{frame}
  \frametitle{Amplification via inventory adjustments}
  Insight from the benchmark framework:
  \begin{itemize}
  \item Assume, \(m_t=d_t+I_{t}-I_{t-1}\), i.e. final sales of imported goods \(d_t\) are equal to final expenditure on imports minus changes in private inventories.
  \item Assume inventories are a constant ratio of final sales, \(\frac{I}{d}\).
    \begin{align*}
      \omega &= \beta_1\hat{\tau}_{ij}+\left(\beta_2-\sigma\right)\left(\hat{p}_i-\hat{P}_j\right) + \left(\beta_3-1\right)\hat{D}_j \\
      \omega &= \frac{I}{d}\sigma\right)\left(\hat{p}_i-\hat{P}_j\right) + \frac{I}{d}\hat{D}_j
    \end{align*}
  \item Inventory adjustment amplifies the sensitivity of imports to changes in
    both relative prices and real final sales.
  \end{itemize}
\end{frame}

\begin{frame}
  \frametitle{Amplification via inventory adjustments}
  The evidence:
  \begin{itemize}
  \item There is little direct evidence outside of the U.S. auto sector.
  \item Imports fell against a less dramatic fall in final sales.
  \item Inventory-to-sales ratios rose and then fell as retailers filled
    orders from inventory.
  \item Other research adopts a partial equilibrium model and calibrates
    with the U.S. auto sector data.
  \item Shocks to the model correspond to the stylized fact that imports
    respond more than final sales do.
  \end{itemize}
\end{frame}

\begin{frame}
  \frametitle{Financial shocks and exports}
  \begin{itemize}
  \item The financial crisis and ensuing credit crunch had effects on trade on
    the supply side.
  \item Organizing the liturature:
    \begin{enumerate}
    \item The financial crisis impeded production, and hence exports, because firms depend on external credit to finance production.
    \item The financial crisis disrupted the financing of international transactions.
    \end{enumerate}
  \item The authors regard international trade finance disruptions as increasing
    the cost of delivering goods to foreign consumers, while holding the price
    of production constant.
  \item Direct, comprehensive data not available.
  \end{itemize}
\end{frame}

\begin{frame}
  \frametitle{Financial shocks and exports}
  Export supply shocks:
  \begin{itemize}
  \item Production at the firm level is a function of demand for output and credit used by the firm.
  \item The credit used is a function of the supply of credit to the firm and
    the firm’s demand for credit, which depends on the demand for output.
  \item Data on credit use are not typically available at the firm or industry level.
  \item The use of credit is endogenous to production decisions.
  \item Changes in demand for output may be correlated with credit disruptions at the firm or sector level.
  \end{itemize}
\end{frame}

\begin{frame}
  \frametitle{Financial shocks and exports}
  The evidence:
  \begin{itemize}
  \item Firm level data indicate that less favorable credit conditions and
    higher liquidity requirements correlated with lower trade during the crisis.
  \item Sector level data suggest exports fell more for sectors high
    external finance dependence.
  \item Matched firm-bank data show bank performance and firm exports linked via export-credit elasticity.
  \end{itemize}
\end{frame}

\begin{frame}
  \frametitle{Financial shocks and exports}
  International trade finance:
  \begin{itemize}
  \item Trade finance refers to financing arrangements for cross-border
    transactions.
  \item Both bank-intermediated and nonbank trade finance are considered.
  \item The financial crisis limited the availability bank-intermediated
    finance.
  \item Limited credit availability may have led firms to hoard liquidity.
  \end{itemize}
\end{frame}

\begin{frame}
  \frametitle{Financial shocks and exports}
  The evidence:
  \begin{itemize}
  \item Markets continued to function despite stresses in bank-intermediated trade finance.
    The decline in bank-intermediated trade finance appears to mostly result from the decline in trade itself, rather than vice versa.
  \item Importers who traded on cash-in-advance terms prior to the crisis decreased their imports by more than those who traded on open account terms.
  \item Trade credit declined less for firms with more short-term debt during
    the crisis.
  \item Firms who were able to substitute toward trade credit experienced smaller declines in sales.
  \end{itemize}
\end{frame}

\begin{frame}
  \frametitle{Recap}
  \begin{itemize*}
  \item Three main proximate causes and/or propagation mechanisms that account for the collapse of international trade:
  \item \textcolor{green}{\cmark} Changes in expenditure: 65-80\%
  \item \textcolor{green}{\cmark} Credit shocks: 15-20\%
  \item \textcolor{green}{\cmark} Inventory adjustments: up to 20\%
  \item \textcolor{red}{\xmark} Trade policy: 0\%
  \end{itemize*}
\end{frame}

\end{document}